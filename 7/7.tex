\documentclass[a4paper, 12pt, fleqn]{article}


\setlength{\textheight}{\paperheight}
\setlength{\topmargin}{-5.4truemm}
\addtolength{\topmargin}{-\headheight}
\addtolength{\topmargin}{-\headsep}
\addtolength{\textheight}{-40truemm}

\setlength{\textwidth}{\paperwidth}
\setlength{\oddsidemargin}{-5.4truemm}
\setlength{\evensidemargin}{-5.4truemm}
\addtolength{\textwidth}{-40truemm}

\usepackage[dvipdfmx]{graphicx}
\usepackage{amsmath}
\usepackage{amsfonts}
\usepackage{mathrsfs}
\usepackage{ascmac}
\usepackage{amsthm}
\usepackage{amssymb}
\usepackage{comment}
\usepackage{latexsym}
\usepackage{comment}
\usepackage{afterpage}

\theoremstyle{definition}
\newtheorem{prb}{Problem}
\title{代数学I宿題(7)}
\author{中野竜之介\ 8310141H}
\begin{document}
\maketitle

\begin{prb}
    $ $
    \begin{enumerate}
        \item $H$ is a normal subgroup of $G$
                $\Leftrightarrow\forall x\in G,\ \forall y \in H,\ xyx^{-1}\in H$.

        \item Define a qutient group $G/H$ as follows.
            \begin{enumerate}
                \item set: $G/H=\{aH|\ a\in G\}$
                \item operation: $aH\cdot bH = (ab)H$ $(a,b\in G)$.
            \end{enumerate}

        \item Let $H$ be any subgroup of $G$. Then for all $x\in G,$ $y\in H$, $xyx^{-1} = xx^{-1}y = ey = y\in H$. Therefore $H$ is a normal subgroup of $G$.
         Since $G$ is a commutative group. Hence $aH\cdot bH=(ab)H=(ba)H=bH\cdot aH$. Therefore $G/H$ is a commutative group.
    \end{enumerate}
\end{prb}


\begin{prb}
    \begin{enumerate}
        \item The table for $A_4$ is as follows.
            \begin{table}[h]
                \begin{tabular}{|c||c|c|c|c|}\hline
                    - & $(1)$ & $(12)(34)$ & $(13)(24)$ & $(14)(23)$ \\ \hline \hline
                    $(1)$ & $(1)$ & $(12)(34)$ & $(13)(24)$ & $(14)(23)$ \\ \hline
                    $(12)(34)$ & $(12)(34)$ & $(1)$ & $(14)(23)$ & $(13)(24)$\\ \hline
                    $(13)(24)$ & $(13)(24)$ & $(14)(23)$ & $(1)$ & $(12)(34)$\\ \hline
                    $(14)(23)$ & $(14)(23)$ & $(13)(24)$ & $(12)(34)$ & $(1)$\\ \hline
                \end{tabular}
            \end{table}

            Because of the table, $A_4$ is a subgroup of $V_4$.

        \item Since $S_4=[(12),(13),(14),(23),(24),(34)]$, the table as follow where $(i,j)$ is $\sigma _i \sigma _j \sigma _i ^{-1}$.
        \begin{table}[h]
            \begin{tabular}{|c||c|c|c|c|}\hline
                - & $(1)$ & $(12)(34)$ & $(13)(24)$ & $(14)(23)$ \\ \hline \hline
                $(12)$ & $(1)$ & $(12)(34)$ & $(14)(23)$ & $(13)(24)$ \\ \hline
                $(13)$ & $(1)$ & $(14)(23)$ & $(13)(24)$ & $(12)(34)$\\ \hline
                $(14)$ & $(1)$ & $(13)(24)$ & $(12)(34)$ & $(14)(23)$\\ \hline
                $(23)$ & $(1)$ & $(13)(24)$ & $(12)(34)$ & $(14)(23)$\\ \hline
                $(24)$ & $(1)$ & $(14)(23)$ & $(13)(24)$ & $(12)(34)$\\ \hline
                $(34)$ & $(1)$ & $(12)(34)$ & $(14)(23)$ & $(13)(24)$\\ \hline
            \end{tabular}
        \end{table}


        \item The table for $A_4 /V_4$ as follow.
        \begin{table}[h]
            \begin{tabular}{|c||c|c|c|}\hline
                - & $V_4$  & $(123)V_4$ & $(132)V_4$ \\ \hline  \hline
                $V_4$      & $V_4$      & $(123)V_4$ & $(132)V_4$ \\ \hline
                $(123)V_4$ & $(123)V_4$ & $(132)V_4$ & $V_4$ \\ \hline
                $(132)V_4$ & $(132)V_4$ & $V_4$      & $(123)V_4$ \\ \hline
            \end{tabular}
        \end{table}

    \end{enumerate}
\end{prb}

\end{document}