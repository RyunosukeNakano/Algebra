\documentclass[a4paper, 12pt, fleqn]{article}


\setlength{\textheight}{\paperheight}
\setlength{\topmargin}{-5.4truemm}
\addtolength{\topmargin}{-\headheight}
\addtolength{\topmargin}{-\headsep}
\addtolength{\textheight}{-40truemm}

\setlength{\textwidth}{\paperwidth}
\setlength{\oddsidemargin}{-5.4truemm}
\setlength{\evensidemargin}{-5.4truemm}
\addtolength{\textwidth}{-40truemm}

\usepackage[dvipdfmx]{graphicx}
\usepackage{amsmath}
\usepackage{amsfonts}
\usepackage{mathrsfs}
\usepackage{ascmac}
\usepackage{amsthm}
\usepackage{amssymb}
\usepackage{comment}
\usepackage{latexsym}
\usepackage{comment}
\usepackage{afterpage}

\theoremstyle{definition}
\newtheorem{prb}{Problem}
\newtheorem{thm}{Theorem}
\newtheorem*{thm*}{Theorem}

\newcommand{\image}[1]{{\rm Im\ }#1}
\newcommand{\kernel}[1]{{\rm Ker\ }#1}



\title{代数学I宿題(10)}
\author{中野竜之介\ 8310141H}
\begin{document}
\maketitle

\begin{prb}
    $ $
    \begin{enumerate}

        \item Let $\pi $ be a plane and $F$ be a regular $n$-sided polygon. Then define $D_n = \{\phi :\pi \to \pi |\ \phi : {\rm motion\ with\ }\phi (F) = F\}$.\\
         Let $\{a_1,...,a_n\}$ denote the vertices of the regular $n$-sided polygon. Since any element in $D_n$ is a motion, for any $\sigma \in D_n$, it will be regarded as a permutation of the vertices. Therefore $D_n$ is a subgroup of $S_n$.

        \item
            \begin{enumerate}
                \item Since $\# D_n = 2n$, $\# D_4 = 8$.
                \item $D_4$ is isomorphic to $[(1234),(24)]$. Since $(1234)(24)=(14)(23)$ and $(24)(1234) = (12)(34)$, $D_4$ is not a non-commutative group.
            \end{enumerate}

        \item $D_3$ is isomorphic to $[(123),(13)]$. And since $S_3 = [(123),(13)]$, $D_3$ is isomorphic to $S_3$.
    \end{enumerate}
\end{prb}



\end{document}