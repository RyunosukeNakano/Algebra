\documentclass[a4paper, 12pt, fleqn, openany]{article}


\setlength{\textheight}{\paperheight}
\setlength{\topmargin}{-5.4truemm}
\addtolength{\topmargin}{-\headheight}
\addtolength{\topmargin}{-\headsep}
\addtolength{\textheight}{-40truemm}

\setlength{\textwidth}{\paperwidth}
\setlength{\oddsidemargin}{-5.4truemm}
\setlength{\evensidemargin}{-5.4truemm}
\addtolength{\textwidth}{-40truemm}

\usepackage[dvipdfmx]{graphicx}
\usepackage{amsmath}
\usepackage{amsfonts}
\usepackage{mathrsfs}%花文字
\usepackage{ascmac}
\usepackage{amsthm}
\usepackage{amssymb}
\usepackage{comment}
\usepackage{latexsym}
\usepackage{comment}
\usepackage{afterpage}

\theoremstyle{definition}
\newtheorem{prb}{Problem}
\title{代数学I宿題(3)}
\author{中野 竜之介\\ 8310141H}
\begin{document}
\maketitle

\begin{prb}
    $ $
    \begin{enumerate}
        \item Since
        \begin{align*}
            \sigma  :& 1 \mapsto 2 \mapsto 3 \mapsto 1,\\
            \tau  :& 1 \mapsto 2 \mapsto 3 \mapsto 4 \mapsto 1,\\\\
                                        &1 \mapsto 2 \mapsto 3\\
            \sigma \tau :               &2 \mapsto 3 \mapsto 1\\
                                        &3 \mapsto 4 \mapsto 4\\
                                        &4 \mapsto 1 \mapsto 2,\\
                        &= 1 \mapsto 3 \mapsto 4 \mapsto 2 \mapsto 1 {\rm\ and}\\\\
            \tau \sigma :               &1 \mapsto 2 \mapsto 3\\
                                        &2 \mapsto 3 \mapsto 4\\
                                        &3 \mapsto 1 \mapsto 2\\
                                        &4 \mapsto 4 \mapsto 1\\
                        &= 1 \mapsto 3 \mapsto 2 \mapsto 4 \mapsto 1,
        \end{align*}
        $\sigma \tau = (1342)$ and $\tau \sigma = (1324)$.

        \item Since
            \[
                \sigma = \left(
                \begin{array}{cccc}
                    1 & 2 & 3 & 4\\
                    2 & 3 & 1 & 4
                \end{array}
                \right),
                \tau = \left(
                \begin{array}{cccc}
                    1 & 2 & 3 & 4\\
                    2 & 3 & 4 & 1
                \end{array}
                \right)
            \]
            \[
                \sigma ^{-1} = \left(
                    \begin{array}{cccc}
                        2 & 3 & 1 & 4\\
                        1 & 2 & 3 & 4
                    \end{array}
                    \right)
                    = \left(
                    \begin{array}{cccc}
                        1 & 2 & 3 & 4\\
                        3 & 1 & 2 & 4
                    \end{array}
                    \right)=(132).
            \]
            \[
                \tau ^{-1} = \left(
                    \begin{array}{cccc}
                        2 & 3 & 4 & 1\\
                        1 & 2 & 3 & 4
                    \end{array}
                    \right)
                    = \left(
                    \begin{array}{cccc}
                        1 & 2 & 3 & 4\\
                        4 & 1 & 2 & 3
                    \end{array}
                    \right)=(1432).
            \]

        \item
            Since
            \begin{align*}
               \tau ^2 :& 1 \mapsto 2 \mapsto 3\\
                        & 2 \mapsto 3 \mapsto 4\\
                        & 3 \mapsto 4 \mapsto 1\\
                        & 4 \mapsto 1 \mapsto 2\\
                        =&1 \mapsto 3,\\
                        &2 \mapsto 4\\
                        =&2 \mapsto 4,\\
                        &1 \mapsto 3,
            \end{align*}
            $\tau ^2 = (13)(24) = (24)(13)$.

        \item Since
            \begin{align*}
                \sigma ^3 :
                        &1 \mapsto 2 \mapsto 3\\
                        &2 \mapsto 3 \mapsto 1\\
                        &3 \mapsto 1 \mapsto 2\\
                        &=1 \mapsto 2 \mapsto 3 \mapsto 1,\\
              \sigma ^3 &= (1)
            \end{align*}
            Since $(13)^2 = (24)^4 = (1)$,
            \begin{align*}
                \tau^4 &= (\tau^2)^2 \\
                       &= (13)(24)(13)(24)\\
                       &= (13)(13)(24)(24) = (1)(1) = (1).
            \end{align*}

        \item
            Since $\sigma = (123)=(13)(12)$ and $\tau = (1234) = (14)(13)(12)$, $\sigma \in A_4$ and $\tau \notin A_4$.
    \end{enumerate}
\end{prb}

\begin{prb}
    Since $(GL(2,\mathbb{R}),\cdot )$ is group isomorphic to $(\hom (\mathbb{R}^2,\mathbb{R}^2),\circ)$, $f_A :\mathbb{R}^2 \to \mathbb{R}^2$ is a bijection and $\hom(\mathbb{R}^2,\mathbb{R}^2)$ is a permutation group, $GL(2,\mathbb{R})$ is a permutation group where $\hom(\mathbb{R}^2,\mathbb{R}^2) = \{f:\mathbb{R}^2 \to \mathbb{R}^2  |\ f{\rm\ is\ a\ bijection\ and\ linear\ maping}\}$.
\end{prb}

\begin{prb}
    $(\mathbb{Q},+),(\mathbb{Z},+)$ are subgroups of $(\mathbb{R},+)$
\end{prb}

\end{document}