\documentclass[a4paper, 12pt, fleqn, openany]{jsarticle}


\setlength{\textheight}{\paperheight}
\setlength{\topmargin}{-5.4truemm}
\addtolength{\topmargin}{-\headheight}
\addtolength{\topmargin}{-\headsep}
\addtolength{\textheight}{-40truemm}

\setlength{\textwidth}{\paperwidth}
\setlength{\oddsidemargin}{-5.4truemm}
\setlength{\evensidemargin}{-5.4truemm}
\addtolength{\textwidth}{-40truemm}

\usepackage[dvipdfmx]{graphicx}
\usepackage{amsmath}
\usepackage{amsfonts}
\usepackage{mathrsfs}%花文字
\usepackage{ascmac}
\usepackage{amsthm}
\usepackage{amssymb}
\usepackage{comment}
\usepackage{latexsym}
\usepackage{comment}
\usepackage{afterpage}
\theoremstyle{definition}
\newtheorem{prb}{Problem}
\title{代数学I宿題(1)}
\author{中野 竜之介\\ 8310141H}
\begin{document}
\maketitle
\begin{prb}
    $ $
        \begin{enumerate}
                \item Binary operation $\circ$ on $G$ is a mapping from $G\times G$ to $G$.
            \item $(G,\circ)$ is a group if and only if it satisfies the following conditions:
                \begin{enumerate}
                    %\item $\forall a,b,c\in G,a\circ(b\circ c) = (a\circ b)\circ c$.
                    \item For all $a,b,c\in G$, $a\circ(b\circ c) = (a\circ b)\circ c$.
                    %a,b,cは複数なのでallを使う.
                    %\item $\exists e \in G$ s.t. $a\circ e = e \circ a = a$.
                    \item There exists $e \in G$ such that, for every $a \in G$, $a\circ e = e \circ a = a$.
                    %\item $\forall a \in G,\exists b \in G$ s.t. $a\circ b = b \circ a = e$.
                    \item For each $a \in G$, there exists $b \in G$ such that $a\circ b = b \circ a = e$.
                \end{enumerate}

            In correct, the conditions (b) and (c) should be united; there exists $e\in G$ such that, for every $a\in G$, $a\circ e = e \circ a = a$ and there exists $b\in G$ such that $a\circ b = b \circ a = e$.\\
                $ $\\
                $(G,\circ)$ is an abelian group if and only if it satisfies the following conditions:
                \begin{enumerate}
                    \item $(G,\circ)$ is a group.
                    %\item $\forall a,b\in G,a\circ b = b \circ a$.
                    \item For all $a,b \in G$, $a\circ b = b \circ a$.
                \end{enumerate}
    \end{enumerate}
\end{prb}

\begin{prb}
    $ $
    \begin{enumerate}
        \item Ring of rational integers $\mathbb{Z}$ is an abelian group.
        \item General linear group $GL(n,\mathbb{C})$ is a non-abelian group.
    \end{enumerate}
\end{prb}


\end{document}